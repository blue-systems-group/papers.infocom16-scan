
\section{The Datasets}
\label{sec:dataset}

To examine the usefulness of \wifi{} scan results collected by smartphones, we
analyze two large scan datasets collected from two smartphone testbeds deployed
in two universities (\ub{} and \nd{}): 5.3M scans from
\PhoneLab{} at \ub{}, and 32M scans from \NetSense{} at \nd{}.
Throughout the paper we refer to these datasets as \textbf{\ubscan{}} and
\textbf{\ndscan{}}, respectively.  Statistics summarizing both datasets are
shown in Table~\ref{tab:stats}. Pending publication of this study, the \ubscan{}
and \ndscan{} datasets will be made available to researchers for further study.

In addition, to compare the client perspective with the AP perspective, we
both (1) obtained access to data generated by the enterprise network software
operating the \ub{} \wifi{} network and (2) performed additional data collection
to address the limitations of that monitoring tool. We refer to these two
datasets as \textbf{\ubap{}} and \textbf{\ubapdetail{}}, and describe them in
more detail in Section~\ref{subsec:cit}.

\begin{table}[t]
  {\small
  %
  \begin{tabularx}{\columnwidth}{Xrr}
    & \PhoneLab{} & \NetSense{}\\\midrule
    Description & \S\ref{sec:phonelab} & \S\ref{sec:netsense} \\
    Identifier & \ubscan{} & \ndscan{} \\ 
    Start & 11/7/2014 & 5/1/2012 \\ 
    End & 4/3/2015 & 3/31/2015 \\ 
    Duration (Days) & 147 & 974 \\ \midrule
    Participants & 254 & 100--125 \\
    Device Type & Nexus~5 & Mixed \\ \midrule
    Scans & \num{5374406} & \num{32564809} \\
    Observed APs & \num{30604} & \num{72001} \\
    Used APs & \num{2742} & \num{2495}\\ \midrule
    \wifi{} Sessions & \num{160886} & \num{149863} \\
    Total Connection Time (Days) & \num{23322} & \num{50969} 
  \end{tabularx}
  %
  \caption{\textbf{Dataset Summary.} Only \wifi{} scans and sessions
  observing the campus network are counted. Used APs refers to the subset of
  total APs that were used by the devices participating in the study. Total
connection time includes only \wifi{} sessions with campus APs.}
  %
  \label{tab:stats}
  \vspace*{-5mm}
}
\end{table}


\subsection{\ubscan{}: \PhoneLab{} \wifi{} Scan Dataset}
\label{sec:phonelab}

\PhoneLab{} is a large scale smartphone platform testbed at \ub{}. Several hundred students,
faculty, and staff carry instrumented LG Nexus~5 smartphones as their primary
device and receive discounted service in return for providing data to smartphone
experiments. \PhoneLab{} participants are distributed across university
departments, making our results representative of the broader campus wireless
network users.

We instrumented the \PhoneLab{} Android Open Source Platform (AOSP)
image provided to participants in August 2014, to log the \wifi{} scan
results naturally generated by the system. Note that while we modified the Android platform to
collect scan results, equivalent data collection can be performed by apps
with the right permissions---as demonstrated by the \ndscan{} dataset described
next.

One scan result contains multiple entries, each corresponding to one nearby
\wifi{} AP observed by the smartphone. The content of one entry includes the
(1) scan timestamp, (2) AP SSID, (3) BSSID, (4) RSSI and (5) AP channel. For
this paper, we are only interested in \wifi{} scans that observe our campus
network, and therefore we remove any scans that do not contain any campus
APs. We also logged \wifi{} connection and disconnection events.

% 06 May 2015 : GWA : AS TODO : Take a look at this.

\subsection{\ndscan{}: \NetSense{} \wifi{} Scan Dataset}
\label{sec:netsense}

The \ndscan{} dataset uses data from the \NetSense{} study
conducted at \nd{}.
\NetSense{}~participants were spatially concentrated in six undergraduate
dormitories, but their demographics (gender, major, and income) were verified
to be representative of the larger undergraduate population.

During the first two years of the study \NetSense{} participants were
provided Nexus~S devices flashed with the Cyanogenmod fork of the AOSP and
running a user-level data collection app. In August~2013, participants were
given the option to continue the study by purchasing their own replacement
handset but continuing to receive free service, and fifty additional
participants were recruited to replace those that chose to quit. From this
point onward, \NetSense{} relied only on the user-level data collection app.

The \NetSense{} data collection app recorded scan results every three
minutes including the (1) scan timestamp, (2) AP SSID, (3) BSSID, and (4) RSSI.
Unlike \ubscan{}, channel information was not recorded.  Beginning in May~2012,
\wifi{} connection events were also logged. For this paper, we utilize only the
data collected from 5/1/2012 to 3/31/2015.

\subsection{Differences Between the Scan Datasets}

% 06 May 2015 : GWA : AS TODO : Take a look at this.

Compared to the \ubscan{} dataset, \NetSense{} devices recorded fewer
sessions per participant day (1.5) than \PhoneLab{} devices (4.3), despite
logging similar numbers of session hours per participant day: 12.5 for
\ndscan{} v. 15.0 for \ubscan{}. We believe that this is largely due to the
difference between the Nexus~S used by \NetSense{} participants during the
first two years of the \ndscan{} dataset and the Nexus~5 used by \PhoneLab{}
participants during the entire \ubscan{} dataset. In particular, the Nexus~S
is known to have poor \wifi{} sensitivity, which may have caused \NetSense{}
devices to initiate sessions more often and end them more quickly. In
addition, \NetSense{} participants are all undergraduate students and spent
various amounts of time on-campus during the three-year study period, leaving
regularly for the summer and on study-abroad programs. In contrast,
\PhoneLab{} participants are mostly staff and would have been on campus
during most of the six-month study.

\subsection{\ub{} \wifi{} Logs and AP Scans}
\label{subsec:cit}

To compare the client- and AP-side perspectives, we first obtained access to the
system logs generated by the Cisco~Prime system~\cite{ciscoprime} used to manage
the campus \wifi{} network of \ub{}. This dataset contains \num{8041604} \wifi{}
sessions from \num{38067} \ub{} campus network users for 44 days from Mar 12 to
Apr 25, 2015. Each record contains these information: 1) the client's MAC
address, 2) the AP's SSID and BSSID, 3) when the \wifi{} session began and
ended, and 4) statistics such as bytes received and transmitted by the AP during
the session. We also obtained an inventory of all \ub{} campus APs, including
their BSSIDs and course-grain location (campus, building and floor).
Collectively we refer to this dataset as \textbf{\ubap{}}.

Unfortunately, the Cisco~Prime interface does not expose all information
collected by the infrastructure network. For example, despite the fact that
\ub{} campus APs clearly perform periodic channel scans for purposes such as
optimizing channel assignment and detecting rogue APs, we were unable to
access the raw scan information---which is either not collected or hidden
behind a proprietary database and not exposed to network administrators.

To address this limitation of the \ubap{} dataset, we augment it with a more
detailed dataset for the 14~APs on the 3rd floor of the CSE department building
at \ub{}. To
reconstruct scan results from these APs, we colocated Nexus~5 smartphones on top
of each AP and configured them to perform channel scans every second for
30~minutes, resulting in \num{1574}~scans per AP on average. We configured a
high scanning rate to try to compensate for the fact that smartphones typically
have less sensitive radio hardware than commodity APs, but there is no way to
perfectly account for these hardware differences, so our dataset should be seen
as an approximation of the scans that could have been collected by the colocated
APs. We refer to this dataset as \textbf{\ubapdetail{}}.
