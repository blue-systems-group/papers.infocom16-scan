\section{Practical Issues}
\label{sec:discussion}

In this section, we discuss several practical issues.

\begin{table*}[t!]
  \centering
  \begin{tabular}{cccccccc}
    \textbf{Case Studies} & \textbf{Device ID} & \textbf{Timestamp} & \textbf{SSID} & \textbf{BSSID} & \textbf{RSSI} & \textbf{Channel} & \textbf{\wifi{} Connection} \\
\toprule
Conflict Graph (\S\ref{subsec:channel}) & $\times$ & $\times$ & $\diamond$ & $\surd$ & $\diamond$ & $\times$ & $\diamond$\\
Spatial Planning (\S\ref{subsec:spatial}) & $\times$ & $\times$ & $\diamond$ & $\surd$ & $\surd$ & $\times$ & $\surd$ \\
\bottomrule

  \end{tabular}
  \caption{\textbf{Summary of Data Required for Each Case Study.} $\times$: Not
  required. $\Diamond$: Optional. $\surd$: Required.}
  \label{tab:privacy}
\end{table*}

\subsection{Privacy}
\label{subsec:privacy}

One particular concern that arises when sharing the channel scan data is
privacy. To address this, we first point out the type of user data that
infrastructure can already collect. Then we discuss the minimum types of channel
scan data required for each case studies. The discussion also serves as our data
anonymization process for releasing the dataset.

In enterprise network environments, since all user devices have to be
authenticated to associate to the APs, the \wifi{} session information, including
user identity, device MAC address and association time, are all recorded by the
management software. Therefore, a complete view of the user's \wifi{} connection
activity are already available at the infrastructure side. Although in the case
studies, we only use such data for ground truth or validation purposes. Even
when user's device is not connected to any APs, the infrastructure can still
learn the presence of the device by sniffing probe packets\footnote{iOS 8
introduce MAC randomization feature in probe packets to preserve user's location
privacy.}.

Next we discuss the minimum types of scan data required to perform each of the
case studies in previous sections. Results are summarized in
Table~\ref{tab:privacy}. First, for all case studies, device identification
information is not utilized in any ways, thus can be stripped off even before
scan results data are uploaded. Second, timestamp information is not needed either, as
long as the multiple scan result entries can be correctly grouped together and
be identified as from one single scan. In
practice, however, certain coarse grained time information, such year and month, can be
kept for post processing convenience. Third, since the channel scan are mainly
used to help monitor central managed networks with predetermined SSIDs, such
information is not required from the scan results. Furthermore, user can choose
only to upload channel scan data when it contains a predefined set of SSIDs, to
avoid revealing information such as home AP SSID. Forth, signal strength
information is necessary in analyzing and prediction the device's association
behavior in spatial planning case study, yet is optional in other two studies.
Additionally, this information can be replaced by an ordering of the APs by signal
strength, eliminating the need for absolute RSSI values. Next, channel
information is not required, since infrastructure has the complete view
of the channel assignments of all APs at any instant. Finally, \wifi{}
connection information is only useful in identifying the association choice from
the scan results in the spatial planning case study. If such information can be
annotated in the scan results before data uploading, \wifi{} connection
information is not required either. For example, device can identify the scans
happened just before a \wifi{} connection event, and mark the corresponding
BSSID as chosen AP in the scan result.

To summarize, only the BSSID information is necessary for the infrastructure to
identify the access points. Other potentially sensitive information can either
be stripped off (e.g., device id, timestamp, channel), or properly anonymized
(e.g., RSSI).

\subsection{Incentive}
\label{subsec:incentive}

We next discuss incentives for user to participate data collection. First, as we
showed earlier in Section~\ref{sec:scan}, smartphones already perform \wifi{}
scans aggressively, thus the natural scan rate already provides a stream of
network measurements with sufficient temporal granularity for monitoring
purpose. By only passively harvesting the scan results, little performance
overhead is incurred in the process. And further steps can be taken to reduce
the energy consumption (e.g., only upload data when charging) and possibility
metered cellular data usage (e.g., only upload through \wifi{} networks).
Additionally, user privacy can be maintained using the anonymization techniques
we discussed in Section~\ref{subsec:privacy}.

Then we look into incentives that encourage user for sharing the data. Note that
in enterprise networks, such data collection consent can be incorporated as part
of the authentication process. For instance, users may be required to install
the data collection app when connecting to enterprise network, and the
installation can then allow the user's other device to connect to the network.
For public \wifi{} service providers such as Boingo~\cite{boingo}, extra Quality
of Service (QoS) or plan discount can be offered to incentivize participation in
data collection.

In conclusion, the measurement overhead can be significantly reduced by passive
data collection and asynchronous delay tolerant data uploading. And effective
incentive mechanisms can be developed to further encourage measurement
participation.

\subsection{Device Heterogeneity}
\label{subsec:heterogeneity}

In the \ubscan{} dataset that is mostly analyzed throughout this paper, all scan
results are collected from devices with the same model (Nexus 5). We now reflect
on whether this hardware homogeneity is necessary, and what the impact of device
heterogeneity would be on each of the case studies.

\wifi{} chipsets and drivers from various manufacturers may have different
polices on the rate of scanning. Yet as we shown earlier in
Section~\ref{sec:scan}, most wireless devices from different vendors all perform
\wifi{} scans quite aggressively.  Therefore, the device heterogeneity shall
have negligible impact on the temporal resolution of the measurements.

On other hand, \wifi{} chipsets do have various RF characteristics, such as
carrier sensing range or antenna sensitivity. For instance, when building
client-perceived conflict graph (\S\ref{subsec:channel}), a more sensitive
device may reveal more conflict edges than those less sensitive ones. And
similarity, devices with different sensitivity levels may disagree on the
offending set of same RAP (\S\ref{subsec:rogue}).

To tolerate such device diversities, the measurement processing logic can either
be aggressive by adopting the superset of observations from different devices, or
be conservative and only consider common conflicts reported by most of the
devices. Additionally, note that relative ordering of the APs by signal strength
is usually consistent for each device, therefore the device heterogeneity has
minimal impact on the spatial planning case study, where only the relative
ordering of the APs instead of the absolute RSSI values is utilized.
