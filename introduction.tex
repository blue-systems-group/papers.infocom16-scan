\section{Introduction}
\label{sec-introduction}

As mobile wireless devices proliferate and the demands they place on wireless
networks continue to grow, monitoring the health and performance of large
scale wireless networks is both essential and increasingly challenging. While
site surveys and infrastructure-side data collection both play important
roles in network monitoring, neither of these approaches can fully and
continuously reflect actual network conditions experienced by the clients the
network is intended to serve---particularly the mobile devices that generate
most wireless traffic. Site surveys are snapshots that are neither spatially nor
temporally representative, and AP-side data collection can only
measure wireless conditions at AP fixed locations, not at the changing
locations of mobile clients. As a result, multiple previous research studies
have demonstrated the value of client-side measurements to improve network
performance~\cite{mishra2005weighted,mishra2006client}, and the 802.11k
amendment~\cite{80211k} provides a mechanism allowing APs to collect this
information from clients to improve radio resource management.

Given the established interest in and emerging support for client-side wireless
network measurement, it is surprising that today we are discarding a
potentially valuable source of client-side information: the large number of
channel scans generated continuously by mobile devices such as smartphones.
These devices naturally perform frequent channel scans to cope with
rapidly-fluctuating network environments caused by mobility. For example,
Android smartphones scan every 15~s when unassociated~\cite{hanover2014cutting}
to search for APs, and continue to scan every 60~s even when already associated in
order to identify better APs as wireless conditions change caused by client mobility,
network interference, or other factors.

Considered as network measurements, channel scans share all the typical benefits
of client-side measurements---such as capturing the real conditions experienced
by end users, providing data that is impossible to gather using APs alone, and
being simpler and more representative than site surveys. In addition, they have
several other unique benefits. First, note that clients spontaneously perform
these channel scans in order to effectively use wireless networks. As a result,
the client overhead required to collect them for monitoring purposes is limited
to the costs of temporary storage (which is minimal) and telemetry (which can be
minimized through delay tolerant and energy neutral data collection). Second, in
many cases channel scans remain useful even after being stripped of information
such as timestamps or device identifiers that could threaten client privacy and
potentially limit the willingness of users to share this information.

All \wifi{} clients perform channel scans, but smartphones have several
advantages that make them uniquely suited to collecting and providing
measurements for monitoring enterprise wireless networks. First, unlike laptops,
smartphones are always powered on, thus scan continuously both during and
between periods of interactive use, providing better temporal resolution than
devices that are regularly powered down or put to sleep. Second, being highly portable, users
are likely to carry smartphones with them most of the time, giving these devices
the ability to observe more of the network than would be seen by stationary or
less-portable devices and making their measurements more representative of all
locations where mobile users might utilize the wireless network. Finally,
smartphone platforms already provide interfaces allowing apps (with appropriate
permissions) to access channel scans, and app marketplaces provide an easy way
to deploy the simple monitoring software required to collect them from billions of
mobile devices. Compared to other mobile devices, smartphones make it (1) easy
to collect (2) an enormous number of channel scans from (3) everywhere that
users might use the wireless networks.

\textbf{But are these measurements actually useful?} Can they contribute
valuable insights in monitoring enterprise wireless networks that would
otherwise be impossible to gain from existing infrastructure-based approaches?
That is the question that we set out to answer in this paper. Our goal is to
determine whether the billions of discarded smartphone channel scans represent a
missed opportunity or redundant information that we can continue to safely
ignore.

To do so, we take a walk on the client side. We utilize two \wifi{} channel scan
datasets collected on two large scale smartphone testbeds (\PhoneLab{}:
\num{5373682}~scans, 254~devices, 5~months; and \NetSense{}:
\num{32564809}~scans, 125~devices, 32~months), described in more detail in
Section~\ref{sec:dataset}. To examine the case for \textit{client-side}
measurement, we ignore analyses that could be performed using measurements from
APs, including APs that utilize extra radio hardware to
continuously perform measurements in parallel with normal client traffic. We
also focus our studies on aspects of network performance and behavior of
interest to network operators, not just scientists, since we anticipate that
large-scale collection of channel scans will only take place if the measurements
are able to provide meaningful operational insights.

Our paper presents two case studies in Section~\ref{sec:case} using one or
both of our channel scan datasets:

\begin{itemize}
  \item \textbf{Section~\ref{subsec:channel}} looks at how the channel conflict
    graph differs from the client perspective, helping operators assess the
    effectiveness of infrastructure-only channel assignment algorithms at
    utilizing available spectrum.

  \item \textbf{Section~\ref{subsec:spatial}} investigates how load can be
    shifted among neighbor APs when a particular AP is not available, both for
    load balancing purpose and to evaluate network redundancy, allowing network
    operators to make better spatial planning decisions.
\end{itemize}

Section~\ref{sec:related} discusses related work before we present and
discuss our conclusions in Section~\ref{sec:conclusion}. We believe that our
case studies demonstrate the value of smartphone channel scans, and will
publish the datasets so that others can extend our analyses or reevaluate
our conclusion.
