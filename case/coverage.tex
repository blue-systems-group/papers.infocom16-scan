\subsection{Coverage}
\label{subsec:coverage}

To begin, we examine the coverage achieved by our dataset. The
139~\PhoneLab{} participants who contributed measurements to our experiment
represent only 0.4\% of the \num{29096} students, faculty, and staff that the
\num{2913}~UB APs are deployed to serve. The smaller the number of
smartphones that can achieve acceptable coverage, the easier the task of
recruiting enough users to adequately monitor the network. However, coverage
means different things depending on the monitoring goals. So below we examine
two different types of coverage: spatiotemporal and periodic AP coverage.

\subsubsection{Spatiotemporal coverage}
\label{subsubsec:coverage}

A useful definition of coverage to a network administrator may encompass how
many APs are observed and how often these APs are seen. We define
\textit{spatiotemporal coverage} as the percentage of all UB APs observed by
participating devices over some time interval. We obtained a list of
\num{2645} active APs from UB network administrators on April 29, 2014. This
list, however, fluctuates over time as APs are added, removed, or replaced.
In fact, our dataset contains \num{2913} UB APs, \num{476} more than the
canonical list provided by UB IT. In the following analysis we ignore these
``ghost'' APs. As a result, the overall spatiotemporal coverage achieved over
the entire five month dataset was $(2913-476)/2645=92\%$.

\begin{figure*}[t]
  \centering
  \includegraphics[width=\textwidth]{./figures/DeviceDayCoverageGraph.pdf}
  \caption{\textbf{Daily and Cumulative Spatiotemporal Coverage.} Mondays on
  the graph are labeled. Coverage shows weekly fluctuations.}
  \label{fig:daily_coverage}
\end{figure*}

Figure~\ref{fig:daily_coverage} shows the daily and cumulative spatiotemporal
coverage achieved during our experiment. Daily coverage exhibits regular
weekly fluctuations, reflecting the fact that \PhoneLab{} participants are
more likely to be on campus during the week than over the weekend---75\% of
experiment participants are faculty and staff, and because UB has a large
percentage of commuter students even students may not be on campus during the
weekend. Cumulative coverage rises quickly in early December, when many new
participants joined the experiment, and then slowly afterward.

We also notice overall reductions in daily coverage during winter
(12/17/2013--1/27/2014) and spring (3/15/2014--3/23/2014) breaks, more so
during winter break when students, faculty, and staff were likely to all be
away from campus. The small increase in the week after spring break may
reflect the fact that many classes held midterm exams during that week.
However, during weekdays when classes were in session we saw roughly 50\% of
all UB APs daily, a remarkable result given the small number of
participants\footnote{Daily coverage is lower until the beginning of December
due to large number of participants that joined the experiment around that
time. See Figure~\ref{fig:daily}.}.

\begin{figure}[t]
  \centering
  \includegraphics[width=0.48\textwidth]{./figures/DeviceSpatialCoverageGraph.pdf}
  \caption{\textbf{Spatiotemporal Coverage v. Number of Devices}}
  \label{fig:device_coverage}
\end{figure}

Next we look at the relationship between spatiotemporal coverage and the
number of participating devices. We were interested to find out if a small
number of participants were responsible for most of the coverage and also
thought that this relationship might help understand how coverage would
change if we were able to recruit more participants. We isolate the week
beginning 2/10/2014 when 86~devices contributed scan results. For $1 \le n
\le 86$, we randomly choose $n$ devices and compute the resulting
spatiotemporal coverage over that week. We repeat the calculation 100 times
for each $n$ and use the average, making our calculations reflect a
representative subset of the devices, not the best subset.
Figure~\ref{fig:device_coverage} shows the resulting relationship, and shows
that a subset of 15~devices can achieve about half of the coverage of the
86~device set.

\begin{figure}[t]
  \centering
  \includegraphics[width=0.48\textwidth]{./figures/APTemporalCoverageGraph.pdf}
  \caption{\textbf{CDF of AP Temporal Coverage.} Experiment durations is 152
  days in total.}
  \label{fig:temporal_coverage}
\end{figure}

Finally, Figure~\ref{fig:temporal_coverage} looks at temporal coverage alone
defined as the percentage of experiment days that each AP was seen.
Approximately 40\% of the campus APs were seen on at least half of the days.

\subsubsection{Periodic coverage}

\begin{figure}[h!]
  \centering
  \begin{subfigure}[t]{\columnwidth}
  \includegraphics[width=\textwidth]{./figures/APPeriodicDayCoverageGraph.pdf}
  \vspace*{-0.3in}
  \caption{All time intervals and APs.}
  \label{fig:periodic:all}
  \end{subfigure}\\
  \begin{subfigure}[t]{\columnwidth}
  \includegraphics[width=\textwidth]{./figures/APPeriodicHourCoverageGraph.pdf}
  \vspace*{-0.3in}
  \caption{Filtered time intervals and APs.}
  \vspace*{0.05in}
  \label{fig:periodic:filtered}
  \end{subfigure}
  \caption{\textbf{Periodic Coverage.} Periodic coverage of
  (\subref{fig:periodic:all}) all APs during all time intervals is poor, but
  (\subref{fig:periodic:filtered}) periodic coverage of commonly-used APs
  over commonly-used times is better.}
  \vspace*{-0.1in}
\end{figure}

Network administrators may also be interested in the rate at which a
smartphone monitoring network can perform repeated measurements of all or
some fraction of network APs. We define \textit{periodic coverage}
parameterized by an interval length $T$ as the percent of APs that are seen
at least once in every interval of length $T$ over a given time period. A
looser definition might consider it acceptable to miss APs in some small
percentage of intervals, but we use the most rigorous definition of periodic
coverage in our analysis.

Figure~\ref{fig:periodic:all} shows that our network achieves poor periodic
coverage, even after excluding the first month of the study when few
participants had joined the experiment: only 20\% of APs were seen even as
rarely as once a week during the experiment. This is due to gaps in coverage
caused by times when participants leave campus: at night, on the weekend, and
during breaks.

However, these also coincide with periods of time when the campus \wifi{}
network is not heavily used and when network monitoring may be less
important. In addition, as we will show later in
Section~\ref{subsubsec:usage}, a small subset of 479~APs (18\%) support 95\%
of the total \wifi{} association time for our 139~participants, making them
both more important to monitor than the rest of the APs and more suitable to
monitor with this set of participants. Figure~\ref{fig:periodic:filtered}
shows the periodic coverage achieved for the 479~heavily-used APs only during
campus working hours (10AM--6PM) on days when classes are in session. With
130~devices, around 30\% of the heavily-used APs can be seen at least once
every day, and around 70\% can be seen every week.
Figure~\ref{fig:periodic:filtered} also shows the relationship between
periodic coverage and the number of devices calculated identically as in
\ref{subsubsec:coverage}.

% 15 Jul 2014 : GWA : Elided for now. Doesn't really fit this section and we
% need some space.
%
% \subsubsection{Interactive coverage}
% 
% \begin{figure*}[t]
%   \centering
%   \includegraphics[width=0.96\textwidth]{./figures/DeviceInteractiveScanGraph.pdf}
%   \caption{\textbf{Per device scan type distribution when $M = 5$.} Where $M$ is
%   margin length in minutes.}
%   \label{fig:interactive}
% \end{figure*}
% 
% \wifi{} condition when user is interactively using the device has an important
% influence on the user's network experience. However, scan frequently in a
% interactive session may hurt the device's network performance. Thus, an
% preferable option would be using the scans \textit{near} (a short time before or
% after) interactive sessions to approximate the \wifi{} conditions during
% interactive session. Since our data collection is passive, we're interested to
% see whether the scans performed by Android can effectively capture the \wifi{}
% conditions of interactive sessions.  
% 
% We divide each scan into three categories depend on the time of scan and the
% screen status.
% \begin{itemize}
%   \item Interactive scan: if scan is performed when screen is on;
%   \item Marginal scan: if scan is performed $M$ minutes before screen is on or
%     $M$ minutes after screen is off;
%   \item Background scan: all other scans.
% \end{itemize}
% 
% We further match each marginal scan with the closest recognized activity within
% 30 seconds. Intuitively, if user remain still after a interactive session, then
% those marginal scans can more actually capture the \wifi{} condition during that
% session. Figure~\ref{fig:interactive} shows the per device percentage of each
% type of scans. In total, around half the scans are performed inside or near a
% interactive session.
