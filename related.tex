\section{Related Works}
\label{sec:related}

Monitoring \wifi{} networks has received a lot of attention over the past
decade. Early
works~\cite{henderson:mobicom2004,meng:mobicom2004,schwab:infocom2004,chen:mccr2010}
used existing AP infrastructure, SNMP logs, and traces collected on the wired
side of the WLAN to analyze \wifi{} traffic. Yeo et
al.~\cite{yeo:wise2004,yeo:witmemo2005} introduced the idea of passive
monitoring using a small number of wireless sniffers. They demonstrated the
feasibility of this approach using synthetic experiments in an isolated
wireless network and discussed the challenges and potential applications. The
same approach was used in a number of
follow-up works~\cite{jardosh:wind2005,jardosh:imc2005}.

The next step was the deployment of large-scale passive wireless monitoring
systems. Jigsaw~\cite{cheng:sigcomm2006} was the first deployed large-scale
monitoring system. 192 sniffers were deployed to report all wireless events
across location, channel, and time to a back-end server. The server used a set
of algorithms and inference techniques to merge and synchronize the traces from
different sniffers into one unified trace which was then used to isolate and
identify the root cause of various performance artifacts, such as data transfer
delays~\cite{cheng:sigcomm2007}.  MAP~\cite{sheng:wicom2008} and its successor,
DIST~\cite{tan:tmc2014}, were security-focused wireless monitoring systems.
Finally, as an alternative to dense sniffer depolyment, wardriving was used
in~\cite{zhou:sigmetrics2013} to construct practical conflict graphs.

The drawback of these systems is the high cost and effort associated with the
dense deployment of static wireless sniffers in order to achieve good
coverage. A few works tried to mitigate the cost by exploiting existing
infrastructure.
DAIR~\cite{bahl:mobisys2006,chan:nsdi2006} uses wireless USB dongles attached
to employee desktop machines instead of deploying
sniffers and uses the collected traces for several applications including
rogue AP detection, helping disconnected
clients, and network performance monitoring. However, DAIR nodes are still
static and suffer from some of the disadvantages of previous solutions.

The idea of using smartphones to monitor wireless networks and/or spectrum has
been exploited in a few recent
works~\cite{nychis:hotwireless2014,nika:hotwireless2014,zhang:hotmobile2015}.
The work in~\cite{nychis:hotwireless2014} uses smartphones to detect and map
heterogeneous networks and devices in home networks. The smartphones
periodically perform measurements and uses them to detect new devices,
determine the impact of one device to another, etc. Finally, the works
in~\cite{nika:hotwireless2014,zhang:hotmobile2015} develop two
crowdsourcing-based RF spectrum monitoring systems using smartphones. The
smartphone is augmented with external hardwares to perform spectrum
measurements---a software defined radio in~\cite{nika:hotwireless2014} and a
frequency translator in~\cite{zhang:hotmobile2015}.

All these works share many of our ideas with respect to the advantages
of a smartphone-based monitoring system over the previously developed
systems using a collection of statically deployed sniffers and the
site-surveys used in today's WLANs. Nonetheless, these works
instrument the smartphones or integrate them with external hardware to
collect specific types of measurements. In contrast, in this paper, we
try to answer the question of whether the billions of \wifi{} channel
scan results collected for free by smartphones can assist in wireless
network monitoring, configuration, or troubleshooting.
