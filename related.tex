\section{Related Works}
\label{sec:related}

Monitoring \wifi{} networks has received a lot of attention over the past
decade. Early works~\cite{chen:mccr2010,
henderson:mobicom2004,meng:mobicom2004,schwab:infocom2004} used existing AP
infrastructure, SNMP logs, and traces collected on the wired side of the WLAN to
analyze \wifi{} traffic. Yeo et al.~\cite{yeo-wise04,yeo:witmemo2005} introduced
the idea of passive monitoring using a small number of wireless sniffers. The
same approach was used in a number of follow-up works~\cite{jardosh:imc2005,
jardosh:wind2005}.

The next step was the deployment of large-scale passive wireless monitoring
systems. Jigsaw~\cite{jigsaw-sigcomm06,jigsaw-sigcomm07} was the first deployed
large-scale monitoring system. 192 sniffers were deployed to report all wireless
events across location, channel, and time to a back-end server, which then
merges different sniffer traces into one unified trace to identify the root
cause of various performance artifacts. MAP~\cite{sheng:wicom2008} and its
successor, DIST~\cite{tan:tmc2014}, were security-focused wireless monitoring
systems. Finally, as an alternative to dense sniffer deployment, wardriving was
used to construct practical conflict graphs~\cite{zhou:sigmetrics2013}.

The drawback of these systems is the high cost and effort associated with the
dense deployment of static wireless sniffers in order to achieve good coverage.
A few works tried to mitigate the cost by exploiting existing infrastructure.
DAIR~\cite{bahl2006enhancing,chan:nsdi2006} uses wireless USB dongles attached
to employee desktop machines and uses the collected traces for several
applications including rogue AP detection, helping disconnected clients, and
network performance monitoring. However, DAIR nodes are still static and suffer
from some of the disadvantages of previous solutions.

The idea of using smartphones to monitor wireless networks and/or spectrum has
been exploited in a few recent
works~\cite{nychis:hotwireless2014,nika:hotwireless2014,zhang:hotmobile2015,shi2014crowdsourcing}.
The work in~\cite{nychis:hotwireless2014} uses smartphones to detect and map
heterogeneous networks and devices in home networks. The smartphones
periodically perform measurements and uses them to detect new devices, determine
the impact of one device to another, etc. The works
in~\cite{nika:hotwireless2014,zhang:hotmobile2015} develop two
crowdsourcing-based RF spectrum monitoring systems using smartphones. The
smartphone is augmented with external hardware to perform spectrum
measurements---a software defined radio in~\cite{nika:hotwireless2014} and a
frequency translator in~\cite{zhang:hotmobile2015}. Pazl~\cite{radu2013pazl} is
a smartphone-based indoor \wifi{} monitoring system. The goal is to develop an
indoor localization system to localize the measurements from different
smartphones, and it is only evaluated via small scale experiments with 5
participants. The closest work to ours is MCNet~\cite{rosen2014mcnet}, a system
that uses smartphones to evaluate the user perceived performance in enterprise
wireless networks. MCNet collects active performance measurements from
smartphones, while we investigate the effectiveness of passive measurements
(channel scans).

All these works share many of our ideas with respect to the advantages of a
smartphone-based monitoring system over the previously developed systems using a
collection of statically deployed sniffers and the site-surveys. Nonetheless,
these works either instrument the smartphones with external
hardware to collect specific types of measurements, or perform active
measurements through deployed softwares. In contrast, in this paper, we try to
answer the question of whether the billions of \wifi{} channel scan results
already being generated by smartphones can assist in wireless network monitoring,
configuration, or troubleshooting.
