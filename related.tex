\section{Related Work}
\label{sec:related}

Monitoring \wifi{} networks has received a lot of attention over the past
decade. Early
works~\cite{balachandran:sigmetrics2002,henderson:mobicom2004,meng:mobicom2004,schwab:infocom2004,chen:mccr2010}
used existing AP infrastructure, SNMP logs, and traces collected on the wired
side of the WLAN to analyze \wifi{} traffic. Yeo et
al.~\cite{yeo:wise2004,yeo:witmemo2005} introduced the idea of passive
monitoring using a small number of wireless sniffers. They demonstrated the
feasibility of this approach using synthetic experiments in an isolated
wireless network and discussed the challenges and potential applications. The
same approach of deploying a small number of sniffers was used in a number of
follow-up works. Jardosh et al. analyze traffic captured by three sniffers at
a large IETF meeting trying to understand link-layer
behavior~\cite{jardosh:wind2005}, and congestion~\cite{jardosh:imc2005} in
802.11b \wifi{} networks. The studies
in~\cite{rodrig:ewind2005,mahajan:sigcomm2006} analyze various performance
characteristics of \wifi{} networks using a trace captured by five sniffers
at a large conference venue. WiserAnalyzer~\cite{chhetri:msn2009} introduces
a set of inference algorithms to determine the relation among wireless nodes
(e.g., non-contending, hidden terminals, carrier sensing) and detect
malicious usage or compliance to the IEEE 802.11 standard.

The next step was the deployment of large-scale passive wireless
monitoring systems consisting of tens or hundreds of sniffers.
Jigsaw~\cite{cheng:sigcomm2006} was the first deployed large-scale
passive monitoring system. The system consists of a 192 sniffers which
report all wireless events across location, channel, and time to a
back-end server. The server uses a set of algorithms and inference
techniques to merge and synchronize the traces from different sniffers
into one unified trace which is then used to isolate and identify the
root cause of various performance artifacts, such as data transfer
delays~\cite{cheng:sigcomm2007}.  MAP~\cite{sheng:wicom2008} was a
security-focused wireless monitoring system.  To enable online
detection of security violations, MAP introduced a number of
optimizations such as frame feature extraction, channel sampling, and
refocusing. The successor of MAP, DIST~\cite{tan:tmc2014} currently
uses 210 sniffers deployed across 11 building at Dartmouth. An
alternative method to dense sniffer deployment is wardriving, often
coupled with calibrated signal propagation models, in order to reduce
the measurement locations. As an example, these two techniques are
used in~\cite{zhou:sigmetrics2013} to construct practical conflict graphs.

The drawback of these systems is the high cost and effort associated with the
dense deployment of static wireless sniffers in order to achieve good
coverage. A few works tried to mitigate the cost by exploiting existing
infrastructure instead of deploying additional hardware.
DAIR~\cite{bahl:mobisys2006,chan:nsdi2006} uses wireless USB dongles attached
to employee desktop machines in an enterprise WLAN instead of deploying
sniffers and uses the collected traces for a number of applications including
rogue AP detection, rogue ad-hoc network detection, helping disconnected
clients, and network performance monitoring. However, DAIR nodes are still
static and suffer from some of the disadvantages of previous solutions.
Different from all previous works, WiFiProfiler~\cite{chandra:mobisys2006} is
a system in which wireless hosts cooperate to diagnose and possibly resolve
network problems in an automated manner, without any infrastructural support,
by communicating with each other in a peer-to-peer fashion. WiFiProfiler
installs custom software on the client to collect detailed network statistics
and exchange data with peers.

\if 0
The closest work to ours is Pazl~\cite{radu:cnsm2013}, a mobile crowdsensing
based indoor \wifi{} monitoring system using measurements from a number of
participant smartphones. The design of Pazl shares many of our ideas with
respect to the advantages of a smartphone-based monitoring system over the
previously developed systems using a collection of statically deployed
sniffers and the site-surveys used in today's WLANs. However, the focus is on
developing an indoor localization system used to localize the measurements
from different smartphones and only evaluates in a small-scale experiment
with five participants. In contrast, our project focuses on \wifi{}
monitoring using a much larger dataset.
\fi

The idea of using smartphones for wireless networks and/or wireless
spectrum monitoring has been exploited in a few recent
works~\cite{radu:cnsm2013,nychis:hotwireless2014,nika:hotwireless2014,zhang:hotmobile2015}. Pazl~\cite{radu:cnsm2013}
is a mobile crowdsensing based indoor \wifi{} monitoring system using
measurements from a number of participant smartphones. However, the
focus of the work is on developing an indoor localization system used
to localize the measurements from different smartphones. Further, the
system is only evaluated in a small-scale experiment with five
participants. The work in~\cite{nychis:hotwireless2014} uses
smartphones to detect and map heterogeneous networks and devices in
home networks. The system includes a training phase during which the
smartphone learns the wireless devices/networks at home, their
locations, their mobility status, and their signal strength
levels. Then, during the monitoring phase, the phone periodically
performs measurements and uses them to detect new devices, determine
the impact of one device to another, and learn spatial/temporal
dynamics. Finally, the works
in~\cite{nika:hotwireless2014,zhang:hotmobile2015} develop two
crowdsourcing-based RF spectrum monitoring systems using
smartphones. The smartphone is connected to an external device which
performs spectrum measurements---a software defined radio
in~\cite{nika:hotwireless2014} and an OpenWrt router hosting a Atheros
9280 card along with a frequency translator
in~\cite{zhang:hotmobile2015}.

All these works share many of our ideas with respect to the advantages
of a smartphone-based monitoring system over the previously developed
systems using a collection of statically deployed sniffers and the
site-surveys used in today's WLANs. Nonetheless, these works
instrument the smartphones or integrate them with external hardware to
collect specific types of measurements. In contrast, in this paper, we
try to answer the question of whether the billions of \wifi{} channel
scan results collected for free by smartphones can assist in wireless
network monitoring. configuration, or troubleshooting.
