\section{Conclusions and Discussion}
\label{sec:conclusion}

Through two extensive case studies, we have demonstrated that smartphone
measurements can contribute unique insights about large scale wireless network
deployments which are difficult or impossible to glean from site surveys,
statically deployed sniffers or infrastructure side logs. Quite simply,
smartphones represent real network users, and their data is representative in
the way that these other sources of measurements cannot be.

To conclude, we address two practical issues in collecting smartphone
scan data: (1) Will smartphone scan data compromise user privacy? (2) What
are the incentives for users to contribute data? 

\begin{comment}
\begin{table*}[t!]
  \centering
  \begin{tabular}{cccccccc}
    \textbf{Case Studies} & \textbf{Device ID} & \textbf{Timestamp} & \textbf{SSID} & \textbf{BSSID} & \textbf{RSSI} & \textbf{Channel} & \textbf{\wifi{} Connection} \\
\toprule
Conflict Graph (\S\ref{subsec:channel}) & $\times$ & $\times$ & $\diamond$ & $\surd$ & $\diamond$ & $\times$ & $\diamond$\\
Spatial Planning (\S\ref{subsec:spatial}) & $\times$ & $\times$ & $\diamond$ & $\surd$ & $\surd$ & $\times$ & $\surd$ \\
\bottomrule


  \end{tabular}
  \caption{\textbf{Summary of Data Required for Each Case Study.} $\times$: Not
  required. $\diamond$: Optional. $\surd$: Required.}
  \label{tab:privacy}
  \vspace*{\aftercaptiongap}
\end{table*}
\end{comment}

\subsection{Privacy}
\label{subsec:privacy}

One particular concern that arises when sharing channel scan data is privacy.
To address this, we first point out the types of user data that the
infrastructure can already collect. Then we discuss the minimum
requirements in terms of channel scan data for each case study.

In enterprise network environments, the \wifi{} session information,
including user's identity, device MAC address and association time, is usually
recorded by the management software. Therefore, a complete view of the user's
\wifi{} connection activity is already available on the infrastructure side.
Even when the user's device is not connected to any AP, the infrastructure
can still detect the presence of the device by sniffing probe
packets. Against this backdrop,
providing channel scan information does not represent an additional loss of
privacy compared to what infrastructure networks can already monitor about
their users.

Next we discuss the minimum requirements in terms of scan data for each of the
case studies in the previous sections.
\begin{enumerate}
    %
  \item For all case studies, device identification information is not
    utilized in any way; thus, it can be properly anonymized even before scan
    results data are uploaded.
    %
  \item Timestamp information is not needed either, as long as the multiple
    scan result entries can be correctly grouped together and be identified as
    from one single scan.
    %
  \item Since the channel scan data are mainly used to help monitor central
    managed networks with predetermined SSIDs, such information is not required
    from the scan results. Furthermore, the user can choose only to upload
    channel scan data when they contain a predefined set of SSIDs, to avoid
    revealing information such as home \wifi{} networks. 
    %
  \item Signal strength information is necessary in analyzing and predicting
    the device's association behavior in the spatial planning case study, yet
    is optional in the other study. Additionally, this information can be
    replaced by an ordering of the APs by signal strength, eliminating the need
    for absolute RSSI values.
    %
  \item Channel information is not required, since the infrastructure has the
    complete knowledge of the channel assignments for all APs at any instant.
    %
  \item \wifi{} connection information is only useful in identifying the
    association choice from the scan results in the spatial planning case
    study. If such information can be annotated in the scan results before data
    uploading, \wifi{} connection information is not required either. For
    example, the device can identify the scans happened just before a \wifi{}
    connection event, and mark the corresponding BSSID as chosen AP in the scan
    result.
    %
\end{enumerate}

To summarize, for the purpose of constructing the conflict graph or performing
spatial planning as in our case studies, only the BSSID information is necessary
for the infrastructure to identify the APs. Other potentially sensitive
information can either be removed or anonymized.

\subsection{Incentives}
\label{subsec:incentive}

We next discuss incentives for users to participate in data collection. First,
smartphones already perform \wifi{} scans aggressively, thus the natural scan
rate already provides a stream of network measurements with sufficient temporal
granularity for monitoring purposes. By only passively harvesting the scan
results, the overhead incurred in the process is kept low. And further steps can
be taken to reduce the energy consumption of date
uploading~\cite{liu2015appatp,zhang5etrain}.  Additionally, user privacy can be
maintained using the anonymization techniques we discussed in
Section~\ref{subsec:privacy}.

Then we look into incentives that encourage users to share the data. Note that
in enterprise networks, such data collection consent can be incorporated as part
of the authentication process. For instance, users may be required to install
the data collection app when connecting to the enterprise network, and the
installation can then allow the user's other devices to connect to the network.
For public \wifi{} service providers such as Boingo, extra Quality
of Service (QoS) or price discounts can be offered to incentivize
participation.

In summary, the measurement overhead can be significantly reduced by passive
data collection and asynchronous delay tolerant data uploading, and effective
incentive mechanisms can encourage measurement participation.

\begin{comment}
  \subsection{Device Heterogeneity}
  \label{subsec:heterogeneity}

  In the \ubscan{} dataset that is mostly analyzed throughout this paper, all scan
  results are collected from devices of the same model (Nexus 5). We now reflect
  on whether this hardware homogeneity is necessary, and what the impact of device
  heterogeneity would be on each of the case studies.

  \wifi{} chipsets and drivers from various manufacturers may have different
  policies on the scanning rate. Yet, as we showed earlier in
  Section~\ref{sec:scan}, most wireless devices from different vendors perform
  \wifi{} scans quite aggressively. Therefore, the device heterogeneity shall
  have negligible impact on the temporal resolution of the measurements.

  On other hand, \wifi{} chipsets do have various RF characteristics, such as
  carrier sensing range or antenna sensitivity. For instance, when building
  client-perceived conflict graph (\S\ref{subsec:channel}), a more sensitive
  device may reveal more conflict edges than a less sensitive one. And
  similarly, devices with different sensitivity levels may disagree on the
  offending set of the same RAP (\S\ref{subsec:rogue}).

  To tolerate such diversity, the measurement post-processing logic can either
  be aggressive by adopting the superset of observations from different
  devices, or be conservative and only consider common conflicts reported by
  most devices. Additionally, note that relative ordering of the APs by
  signal strength is usually consistent for each device, therefore the device
  heterogeneity has minimal impact on the spatial planning case study, where
  only the relative ordering of the APs instead of the absolute RSSI values is
  utilized.


\subsection{Future Works}

Inspired by our results, we are developing a lightweight scan collection app
and set of analysis tools that we will make available on the Google Play
store and to network operators interested in using scans to better
understand their enterprise \wifi{} networks. We also note that the concepts and
techniques developed may also be extended to cellular networks to achieve
similar benefits.
\end{comment}
